% Preamble
% ---
\documentclass[10pt, a4paper, leqno]{article}
\providecommand{\tightlist}{%
  \setlength{\itemsep}{0pt}\setlength{\parskip}{0pt}}

% Packages
% ---

\usepackage{amsmath} % Advanced math typesetting
\usepackage{amssymb}
\usepackage[utf8]{inputenc} % Unicode support {Umlauts etc.}
\usepackage{hyperref} % Add a link to your document
\usepackage{graphicx}
\usepackage{float}


\usepackage[a4paper, margin=1em]{geometry}

\usepackage{multicol}
\setlength{\columnsep}{2em}
\setlength{\columnseprule}{1pt} 

\usepackage{titlesec}

\titlespacing\section{0pt}{1pt plus 1pt minus 2pt}{0pt plus 2pt minus 2pt}
\titlespacing\subsection{0pt}{1pt plus 1pt minus 2pt}{0pt plus 2pt minus 2pt}
\titlespacing\subsubsection{0pt}{1pt plus 1pt minus 2pt}{0pt plus 2pt minus 2pt}

\usepackage{etoolbox}
\makeatletter
\preto{\@verbatim}{\topsep=0pt \partopsep=0pt }
\makeatother
 
\begin{document}



\begin{multicols}{2}

\section {System Calls}

    \subsection{Pipe}

    returns: 0 success, -1 error

    puts FDs of pipe in the argument array

    \begin{verbatim}
    int pipe(int pipeFD[2]);
    \end{verbatim}

    \subsection{Dup}

    \begin{verbatim}
    int newFD = dup(int oldFD);
    int newFD = dup2(int oldFD, int newFD);
    \end{verbatim}
    dup2 copies oldFD onto newFD, so the fileD at newFD becomes oldFD

    \subsection {Fork} 

    \texttt{pid\_t fork()}

    returns pid to parent, and 0 to child. 


    \subsection{Exec}

    All exec functions replace the call stack. 
    The first element of argv must be the filename to execute. 

    \begin{verbatim}
   int execv(const char *path, char *const argv[]);
   int execvp(const char *file, char *const argv[]);
   int execvpe(const char *file, char *const argv[],
   char *const envp[]);
\end{verbatim}
    \subsubsection {Example}
    \begin{verbatim}
    dup2(hubToPlayer[0], 0);
    dup2(playerToHub[1], 1);
    dup2(devNull, 2);

    char playerIDArg[ARG_SIZE];
    sprintf(playerIDArg, "%d", i);

    execlp(playerExecutables[i], 
        playerExecutables[i],
        numPlayersArg, playerIDArg,
        thresholdArg, handArg, (char*) 0);
\end{verbatim}

    \subsection {Wait}

    \begin{verbatim}
int exitStatus;
pid_t wait(int *stat_loc);
// possible options: WNOHANG
int options = 0;
//pid_t waitpid(pid_t pid, &exitStatus, int options);
pid_t waitpid(451, &exitStatus, 0);
int status = WEXITSTATUS(exitStatus);
bool WIFEEXITED(exitStatus); 
    // true if exit() was called by child
int WEXITSTATUS(exitStatus);
    // The value given to exit() by child
bool WIFSIGNALED(exitStatus); 
    //true if exit without exit() 
int WTERMSIG(exitStatus); 
    // the signal that killed child
\end{verbatim}

\subsection {Signals}

1 HUP, 2 INT, 9 KILL, 11 SEGV, 13 PIPE

\begin{verbatim}
static void sighup_handler(int signum);
int main() {
    struct sigaction saHup;
    saHup.sa_handler = sighup_handler;
    saHup.sa_flags = SA_RESTART;
    sigaction(SIGHUP, &saHup, NULL);
}
int main() {
    sigset_t signalMask;
    sigemptyset(&signalMask);
    sigaddset(&signalMask, SIGPIPE);
    pthread_sigmask(SIG_BLOCK, &signalMask, 0);
}
void *handler(void *) {
    sigset_t waiton; // setup sigset
    while (!sigwait(&waiton)) {
        // do things when receiving signal
    }
}
\end{verbatim}

\subsection {PThread}

\begin{verbatim}
pthread_mutex_init(pthread_mutex_t);
int pthread_create(&threadID, attr, 
        void*(*func)(void*), void*arg);

void pthread_exit(void *retval);
int pthread_join(threadID, void**retval);

sem_init(sem_t *sem, 0, initialVal);
sem_post(sem_t); sem_wait(sem_t*); 
sem_trywait(sem_t*);
\end{verbatim}

\subsection {stdio}

\texttt{FILE *fdopen(int FD, char *mode)}

\texttt{int sscanf(string, format, ...)}

\texttt{char *fgets(char *retstr, int maxchars, FILE*)}

Reads until eof or newline, terminating newline is stored. 

\section {Networks}

\subsection {DNS}
Over UDP
\subsection {UDP}
Discrete \emph{Datagrams}, no handshake and verification. Messages have a mx size, no 
delivery acknowledgement messages (unless you implement it on top). 
\subsection {TCP}
Bi-Directional Connection oriented. Provides reliability (keep sending until you get an ACK).

Segments

ACK and NAK Messages. 

\subsection {Notation}

CIDR: Set host bits to 0 \texttt{/numnetbits}. Netmask: Set all network bits to 1, all host bits to 0.

\subsection {Special Addresses}
Gateway: All host bits 0. Broadcast: All host bits 1.
\subsubsection {Non-routable ips}

\begin{itemize}
\tightlist
\item 10.0.0.0 / 8
\item 192.168.0.0 / 16
\item 127.0.0.0 / 8 Loopback
\item 169.254.0.0 / 16 ONLY used for fallback when DHCP failed
\end{itemize}

\subsection {NAT = Network Address Translation}

Rewrites source IP for TCP requests, and keeps track using incoming port-outgoing port on each side of the network. 

\subsection {Layers}

\begin{enumerate}
\tightlist
\item Physical
    \begin{itemize}
    \tightlist
        \item Wires
    \end{itemize}
\item DataLink
    \begin{itemize}
    \tightlist
        \item Ethernet
        \item MAC Addressing (48bit)
    \end{itemize}
\item Network
    \begin{itemize}
    \tightlist
        \item IP: Internet Protocol
        \item IPv4/v6 Addressing (v4 32 bit)
    \end{itemize}
\item Transport
    \begin{itemize}
    \tightlist
        \item UDP/TCP
        \item Port numbers
    \end{itemize}
\item Application
    \begin{itemize}
    \tightlist
        \item HTTP + HTML
        \item GET/POST
    \end{itemize}
\end{enumerate}


\section {Memory}

\begin{verbatim}
void *malloc(size_t size);
void free(void *ptr);
void *calloc(size_t nmemb, size_t size);
void *realloc(void *ptr, size_t size);
\end{verbatim}

\subsection {TLB = Translation Look-aside Buffer}

Hardware memory page$\to$frame cache, global.

\subsection {Disk fragmentation}
Internal: There is unused space in allocated blocks (because files are smaller than blocksize).
External: There is no large contiguous free space; small files are created and deleted everywhere, 
bad indexing for expanding files. 

\subsection {C Types}

\begin{itemize}
    \tightlist
\item int: $\geq$ 16bits
\item INT modifiers: signed,unsigned,short,long 
\item \texttt{long} alone implies int $\geq$ 32bits
\item long long: $\geq$ 64 bit int. 
\item float: single precision float 32bits
\item double: double precision float 64bits
\item long double: extended precision float 96/128bit
\end{itemize}


\section {Bash}

Wildcards: \texttt{*} and \texttt{?}

\begin{itemize}
    \tightlist
    \item[\texttt{uniq:}] comp consecutive lines [-d (print duplicates) –c (prefix line with count of
occurences) –i (ignore case) –u (print unique lines) –sN (skip first N chars) –wN (only count
first N chars)
\item [\texttt{sort:}] [-r] reverse order [-k] key
\item [\texttt{kill:}] -s SIGNAME
\item \texttt{head / tail} -n number of lines
\item [\texttt{grep:}] -v invert, regex: \texttt{\^} beginning of line, \$ end of line, . any character, * match 0 or more times, [] match any one character inside []. 
\item [\texttt{tr:}] -s 'c', combine all ocurences of 'c' in line into one, -d 'c' delete all occurences of c
\item [\texttt{pgrep}]: -x exact match, -c count, -u \$USER, user
\end{itemize}

\begin{verbatim} 
cut -f$fieldNum -d$delimiterchar 
chmod $mode $filename
ln -s $linkname $targetname 

$ ps -ef
UID        PID  PPID   C STIME TTY          TIME CMD
user    procID parenID                      age  name

$ ls -li | tr -s ' '
4719745 -rwxrw-r-- 1 root root 2048 Jan 13 07:11 bob 
$ ls -d 
.
$ ls -1
alice
bob
\end{verbatim}

\begin{itemize}
    \tightlist
    \item inode number
    \item  \texttt{[-/d/l]uuugggooo} File, symlink, or directory, then permissions 
        \item Number of hardlinks
        \item owner name
        \item owner group
        \item file size (bytes)
        \item modification time and file name
        \item symlink info
\end{itemize}

\begin{verbatim}
 ________________
< GNU can do it! >
 ----------------
  \
   \     
    \     (____)
           (oo)------___
           (__)\        \/\
               ||----\ |
               ||     ||

\end{verbatim}

\end{multicols}

\end{document}

